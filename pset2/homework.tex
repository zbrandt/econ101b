\documentclass{article}

\usepackage{fancyhdr}
\usepackage{extramarks}
\usepackage{amsmath}
\usepackage{amsthm}
\usepackage{amsfonts}
\usepackage{tikz}
\usepackage[plain]{algorithm}
\usepackage{algpseudocode}
\usepackage{pgfplots}

\usetikzlibrary{automata,positioning}

%
% Basic Document Settings
%

\topmargin=-0.45in
\evensidemargin=0in
\oddsidemargin=0in
\textwidth=6.5in
\textheight=9.0in
\headsep=0.25in

\linespread{1.1}

\pagestyle{fancy}
\lhead{\hmwkAuthorName}
\chead{}
\rhead{\hmwkClass\ (\hmwkClassInstructor\ \hmwkClassTime): \hmwkTitle}
\lfoot{\lastxmark}
\cfoot{\thepage}

\renewcommand\headrulewidth{0.4pt}
\renewcommand\footrulewidth{0.4pt}

\setlength\parindent{0pt}

\pgfplotsset{compat=1.18}

%
% Create Problem Sections
%

\newcommand{\enterProblemHeader}[1]{
    \nobreak\extramarks{}{Problem \arabic{#1} continued on next page\ldots}\nobreak{}
    \nobreak\extramarks{Problem \arabic{#1} (continued)}{Problem continued on next page\ldots}\nobreak{}
}

\newcommand{\exitProblemHeader}[1]{
    \nobreak\extramarks{Problem \arabic{#1} (continued)}{Problem \arabic{#1} continued on next page\ldots}\nobreak{}
    \stepcounter{#1}
    \nobreak\extramarks{Problem \arabic{#1}}{}\nobreak{}
}

\setcounter{secnumdepth}{0}
\newcounter{partCounter}
\newcounter{homeworkProblemCounter}
\setcounter{homeworkProblemCounter}{1}
\nobreak\extramarks{Problem \arabic{homeworkProblemCounter}}{}\nobreak{}

%
% Homework Problem Environment
%
% This environment takes an optional argument. When given, it will adjust the
% problem counter. This is useful for when the problems given for your
% assignment aren't sequential. See the last 3 problems of this template for an
% example.
%
\newenvironment{homeworkProblem}[1][-1]{
    \ifnum#1>0
        \setcounter{homeworkProblemCounter}{#1}
    \fi
    \section{Problem }
    \setcounter{partCounter}{1}
    \enterProblemHeader{homeworkProblemCounter}
}{
    \exitProblemHeader{homeworkProblemCounter}
}

%
% Homework Details
%   - Title
%   - Due date
%   - Class
%   - Section/Time
%   - Instructor
%   - Author
%

\newcommand{\hmwkTitle}{Problem Set\ \#2}
\newcommand{\hmwkDueDate}{October 7, 2024}
\newcommand{\hmwkClass}{Macroeconomics}
\newcommand{\hmwkClassTime}{Section 102}
\newcommand{\hmwkClassInstructor}{Prof. Barnichon}
\newcommand{\hmwkAuthorName}{\textbf{Zachary Brandt}}

%
% Title Page
%

\title{
    \vspace{2in}
    \textmd{\textbf{\hmwkClass:\ \hmwkTitle}}\\
    \normalsize\vspace{0.1in}\small{Due\ on\ \hmwkDueDate\ at 2:00pm}\\
    \vspace{0.1in}\large{\textit{\hmwkClassInstructor\ \hmwkClassTime}}
    \vspace{3in}
}

\author{\hmwkAuthorName}
\date{}

\renewcommand{\part}[1]{\textbf{\large Part \Alph{partCounter}}\stepcounter{partCounter}\\}

%
% Various Helper Commands
%

% Useful for algorithms
\newcommand{\alg}[1]{\textsc{\bfseries \footnotesize #1}}

% For derivatives
\newcommand{\deriv}[1]{\frac{\mathrm{d}}{\mathrm{d}x} (#1)}

% For partial derivatives
\newcommand{\pderiv}[2]{\frac{\partial}{\partial #1} (#2)}

% Integral dx
\newcommand{\dx}{\mathrm{d}x}

% Alias for the Solution section header
\newcommand{\solution}{\textbf{\large Solution}}

% Probability commands: Expectation, Variance, Covariance, Bias
\newcommand{\E}{\mathrm{E}}
\newcommand{\Var}{\mathrm{Var}}
\newcommand{\Cov}{\mathrm{Cov}}
\newcommand{\Bias}{\mathrm{Bias}}

\begin{document}

\pgfplotsset{
    standard/.style={
        axis line style = thick,
        trig format=rad,
        enlargelimits,
        axis x line=middle,
        axis y line=middle,
        enlarge x limits=0.15,
        enlarge y limits=0.15,
        every axis x label/.style={at={(current axis.right of origin)}, anchor=north west},
        every axis y label/.style={at={(current axis.above origin)},anchor=south east},
        % grid=both,
        ticklabel style={font=\large, fill=white}
    }
}

\maketitle

\pagebreak

\begin{homeworkProblem}
    Fiscal Stimulus in a Neoclassical Model. The econonomic model that we have been developing up
    until now in the course has been ``Neoclassical'' in the sense that all markets have been assumed
    to be competitive and we have abstracted from all ``market failures.'' In this question, we explore
    the implications of fiscal stimuls - i.e., increases in government purchases - on output and
    consumption in this Neoclassical model. Later in the class, we will study fiscal stimulus in a 
    Keynesian model.
    \\ \\
    Consider a Robinson Crusoe economy (i.e., an economy populated by a large number of identical
    households). In such an economy, all households will do the same thing since everyone is identical.
    We can therefore represent the whole household side of the economy by on e``representative 
    consumer,'' which we refere to as Robinson Crusoe.
    \\ \\
    One somewhat tricky aspect of writing the model this way is that we will be using the same
    symbols to represent both individual variables and teh corresponding aggregate variables. For
    example, we will use the symbol $C$ to represent individual consumption but also to represent
    aggregate consumption. The same will be true for hours worked $H$, government spending $G$, and
    taxes $T$. It is important to keep in mind when solving the problem that from the individual's point
    of view certain variables are exogenous - i.e., taken as given. For example, each household doesn't
    take into account how a change in consumption and hours worked will affect taxes and government
    spending because from its perspective any change in its behavior has a trivial effect on the
    aggregate and thus a trivial effect on taxes and government spending. 
    \\ \\
    Supose Robinson Crusoe's preferences can be represented by the following utility function:

    \[
        \log(C) + \psi log(1-H) + \theta log(G)
    \]
    
    Here $C$ denotes consumption and $H$ denotes hours worked as in the models we have seen earlier
    in the class. The new element is $G$, which represents government purchases. We entertain the
    possibility that Robinson Crusoe may value the things that the government purchases. This is why
    $G$ shows up in Robinson Crusoe's utility function. The degree to which Robinson Crusoe values
    government purchases is governed by the parameter $\theta$. Suppose Robinson Crusoe's budget
    constraint is $C=wH-T$, where $w$ denotes the wage rate and $T$ denotes the lump sum taxes paid by
    Robinson Crusoe. Assume for simplicity that the government runs a balanced budget, i.e., that $G =
    T$. Notice, also, that the resource constraint in this economy implies that $Y=C+G$, i.e.,
    consumption plus government purchases cannot exceed the amount of output produced $Y$. In this
    model, we will treat $C$, $H$, and $Y$ as endogenous variables. All other variables - including $G$ and $T$
    - are considered exogenous.
    \\

    \pagebreak

    \part

    Derive Robinson Crusoe's labor supply curve. (Hint: Since $G$ is exogenous,
    Robinson Crusoe treats it as a constant.)
    \\

    \solution
    \\
    Robinson Crusoe will attempt to maximize his utility function subject to his budget constraint.
    To do so he will balance the marginal utility of consumption with the marginal disutility of work.
    First, we can substitute the budget constraint into the utility function to eliminate $C$: 
    
    \[
        U = \log(wH-T) + \psi \log(1-H) + \theta \log(G)
    \]
    
    Now we can see both the utility of consumption and of leisure are functions of hours worked $H$
    (utility from government purchases is exogenous and invariable with respect to $H$ from the individual's
    perspective).
    We can now differentiate with respect to $H$ and set the result equal to zero to find the optimal
    amount of hours worked $H$ that maximizes Robinson Crusoe's utility. This will define the labor
    supply curve for this economy. 

    \[
        \begin{split}
            \frac{\partial U}{\partial H} &= \frac{w}{wH-T} - \psi \frac{1}{1-H}
            \\
            0 &= \frac{w}{wH-T} - \psi \frac{1}{1-H} 
            \\
            \psi \frac{1}{1-H} &= w \frac{1}{wH-T}
        \end{split}
    \]

    The above equation solves for hours worked at each wage rate $w$ by balancing the marginal 
    disutility of an extra hour work on the left hand side, and the marginal value of consumption 
    from an extra hour worked on the right. 


    \pagebreak

    \part
    
    Suppose the production function in the economy is $Y=AL$ and the wage is thus
    given by $w = A$ as in Problem Set 1. Use this equation, Robinson Crusoe's labor supply curve, the
    economy's resource constraint and/or the balanced budget equation to solve for output in terms of
    only $G$, $A$, and $\phi$. (Hint: Recall that $N=1$.)
    \\ \\ 
    \solution

    We can substitute the wage rate $w = A$ into the labor supply curve we derived in part (a) to find
    the hours worked per person in the economy. We can then substitute this expression for hours worked
    into the resource constraint $Y = C + G$ to find output in terms of only $G$, $A$, and $\phi$.

    \begin{minipage}{0.5\textwidth}
        \[
            \begin{split}
                \psi \frac{1}{1-H} &= w \frac{1}{wH-T}
                \\
                \psi \frac{1}{1-H} &= A \frac{1}{AH-T}
                \\
                \psi (AH-T) &= A (1-H)
                \\
                \psi AH - \psi T &= A - AH
                \\
                \psi AH + AH &= A + \psi T
                \\
                H (\psi A + A) &= A+\psi T
                \\
                H &= \frac{A+\psi T}{\psi A + A}
            \end{split}
        \]
    \end{minipage}
    \begin{minipage}{0.5\textwidth}
        \[
            \begin{split}
                Y &= C + g
                \\
                Y &= wH - T + G
                \\
                Y &= AH - G + G
                \\
                Y &= AH
                \\
                Y &= A \frac{A+\psi T}{\psi A + A}
                \\
                Y &= A \frac{A + \psi G}{A \psi + A}
                \\
                Y &= \frac{A + \psi G}{\psi + 1}
            \end{split}
        \]
    \end{minipage}

    \pagebreak

    \part

    The government purchases multiplier is defined as the number of dollars that output 
    rises by when government purchases rise by one dollar. (You can assume for simplicity that all the 
    endogenous variables are denoted in dollars.) What is the government pruchases multipler in this
    economy? If $\psi > 0$, what is the range of values that the government purchases multiplier can take?
    \\ \\

    \solution

    The government purchases multiplier is the derivative of output with respect to government purchases.
    We can take the derivative of the output function we derived in part (b) with respect to $G$ to find the
    government purchases multiplier.

    \[
        \begin{split}
            \frac{\partial Y}{\partial G} &= \frac{\partial}{\partial G} \left( \frac{A + \psi G}{\psi + 1} \right)
            \\
            &= \frac{\psi}{\psi + 1}
        \end{split}
    \]

    The government purchases multiplier is $\frac{\psi}{\psi + 1}$. If $\psi > 0$, the government purchases
    multiplier can take on any value in the range $(0, 1)$.

    \pagebreak

    \part

    Solve for consumption in terms of on only $G$, $A$, and $\psi$. Briefly comment on how an
    increase in government purchases affects consumption. 
    \\

    \solution

    We can substitute the hours worked expression we derived in part (b) into the budget constraint $C = wH - T$
    to find consumption in terms of only $G$, $A$, and $\psi$.

    \[
        \begin{split}
            C &= wH - T
            \\
            C &= A \frac{A + \psi T}{\psi A + A} - G
            \\
            C &= \frac{A + \psi G}{\psi + 1} - G
            \\
            C &= \frac{A + \psi G}{\psi + 1} - \frac{G(\psi + 1)}{\psi + 1}
            \\
            C &= \frac{A + \psi G - G(\psi + 1)}{\psi + 1}
            \\
            C &= \frac{A + \psi G - G\psi - G}{\psi + 1}
            \\
            C &= \frac{A - G}{\psi + 1}
        \end{split}
    \]
    
    From this substitution we can see that an increase in government purchases will result in a decrease in
    consumption, all else held equal. 

    \pagebreak

    \part

    In one paragraph, discuss whether an increase in government purchases makes 
    Robinson Crusoe better or worse off. In particular, commment on whether Robinson Crusoe is made
    better off in the case where he does not value the things the government purchases, i.e., if $\theta = 0$.
    \\

    \solution 

    % maths explanation with offset from hours worked (function of T, G)

    To determine whether an increase in government purchases makes Robinson Crusoe better or worse off, we can
    substitute the consumption function from part (d) and the expression for hours worked from part (b) into
    Robinson Crusoe's utility function. We can then differentiate the utility function with respect to $G$ to
    determine the effect of an increase in government purchases on Robinson Crusoe's utility.

    \[
        \begin{split}
            U &= \log(C) + \psi \log(1-H) + \theta \log(G)
            \\
            U &= \log \left( \frac{A - G}{\psi + 1} \right) + \psi \log \left( 1 - \frac{A + \psi G}{\psi A + A} \right) \;\;\; \text{if} \; \theta = 0
            \\
            \frac{\partial U}{\partial G} &= \frac{1}{\frac{A - G}{\psi + 1}} \left( -\frac{1}{\psi + 1} \right) + \frac{\psi}{1 - \frac{A + \psi G}{\psi A + A}} \left( -\frac{\psi}{\psi A + A} \right)
            \\
            &= -\frac{1}{A - G} + \frac{\psi}{\frac{\psi A + A}{\psi A + A} - \frac{A + \psi G}{\psi A + A}} \left( -\frac{\psi}{\psi A + A} \right)
            \\
            &= -\frac{1}{A - G} + \frac{\psi}{\frac{\psi A - \psi G}{\psi A + A}} \left( -\frac{\psi}{\psi A + A} \right)
            \\
            &= -\frac{1}{A - G} + \frac{1}{\frac{A - G}{\psi A + A}} \left( -\frac{\psi}{\psi A + A} \right)
            \\
            &= -\frac{1}{A - G} + \frac{\psi A + A}{A - G} \left( -\frac{\psi}{\psi A + A} \right)
            \\
            &= -\frac{1}{A - G} - \frac{\psi}{A - G}
            \\
            &= -\frac{1 + \psi}{A - G}
        \end{split}  
    \]
    \\
    From the above expression we can see that an increase in government purchases will make Robinson Crusoe worse off
    if $\theta = 0$ and $\psi > -1$. So if $\psi > 0$ like before, the marginal utility of government purchases is negative. 
    Even before differentiation it's clear to see that $U$ will decrease when $G$ increases. For the log of consumption,
    any increase in $G$ will decrease the log expression overall, meaning decreased utility. For the log of 1 minus hours
    worked, any increase in $G$ will increase the term that is substract from 1, again decreasing the log expression overall
    and decreasing utility.
    \\ \\
    All together, the decrease in consumption from increased government purchases, and the ensuing decrease in utility, is not offset 
    by a decrease in hours worked in the case where Robinson Crusoe values nothing the government purchased.
    
    \pagebreak

\end{homeworkProblem}

\pagebreak


\end{document}
