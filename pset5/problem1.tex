\begin{homeworkProblem}[1]
    In this problem, we propose to explore two famous episodes when the price 
    level fell substantially. One was in England between 1250 and 1490. The 
    other episode was in the United States between 1879 and 1896.
    \\ \\
    One commonly argued reason why the price level fell during these periods 
    was that the economies in question were growing and this was putting downward
    pressure on prices as the supply of money was not growing fast enough to
    offset the growth in the economy. In this problem, we use an extension of
    the medieval economy model to formalize this argument.
    \\ \\
    Consider a version of the medieval economy model in which the desired level
    of output may change over time. Since it may change over time, we denote it
    as $Y_t^*$ rather than $Y^*$. You can think of the reasons why the desired 
    level of output can change over time being either that the size of the 
    population changes over time or that the level of technology changes over 
    time so each person can produce more output per hour worked.
    \\ \\
    Since the desired level of output can change, the price setting equation
    becomes:
    
    \begin{flalign*}
        & \quad \quad \text{PS:} \quad \quad \: \log P_{t+1} - \log P_t = \theta (\log Y_t - \log Y_t^*) &\\
    \end{flalign*}
    
    Suppose the money market equilibrium condition is the same as before:
    
    \begin{flalign*}
        & \quad \quad \text{MME:} \quad \log M_t + \log V = \log P_t + \log Y_t &\\
    \end{flalign*}
    
    A) Suppose for simplicity that $\log Y_t^*$ is a constant equal to 0. Solve
    for the steady state level of output and steady state price level when $\log M_t 
    = 0$ and $\log V=0$. 
    \\ \\
    B) Now suppose that the growth rate of the desired level of ouput rises
    such that $\Delta \log Y_t^* = 0.02$ but the money supply remains fixed
    at $\log M_t = 0$ and velocity remains fixed at $\log V=0$. Solve for the
    steady state inflation rate (i.e., steady state value of $\Delta \log P_t$)
    and the steady state output gap (i.e., the steady state value of $\log Y_t
    - \log Y_t^*$). \textit{Hint}: You should guess that $\Delta \log P_t$ is 
    constant in this steady state (i.e., the change in the log price level is
    constant). Then, subtract the time $t - 1$ version of the PS equation above 
    from the time $t$ version of this equation. The resulting equation should 
    help you solve the problem.
    \\ \\
    C) Starting from the steady state you calculated in part A), suppose that
    the growth rate of the desired level of output rises such that $\Delta \log 
    Y_t^* = 0.02$. Solve for the evolution of $\log P_t$ and $\log Y_t$ over the
    next 20 periods. Assume that $\theta = 0.5$. Again assume that the 
    money supply remains fixed at $\log M_t = 0$ and velocity remains fixed at
    $\log V = 0$. Graph the result. 
    
    \pagebreak
    
    \part
    
    Suppose for simplicity that $\log Y_t^*$ is a constant equal to 0. Solve
    for the steady state level of output and steady state price level when $\log M_t 
    = 0$ and $\log V=0$. 
    \\ \\
    \solution

    To solve for steady-state output, I will use the price setting equation where
    variables no longer change over time and substitue in 0 for $\log Y_t^*$

    \[
        \begin{split}
            \log P_{t+1} - \log P_{P_t} &= \theta (\log Y_t - \log Y_t^*)
            \\
            \log P - \log P &= \theta (\log Y - 0)
            \\
            0 &= \theta (\log Y)
            \\
            0 &= \log Y
            \\
            Y &= 1
        \end{split}
    \]

    For the steady-state price level, I will use the money market equilibrium 
    condition where again variables don't change over time and substitute in 
    0 for $\log M_t$, $\log V$ and steady-state output

    \[
        \begin{split}
            \log M_t + \log V &= \log P_t + \log Y_t
            \\
            0 + 0 &= \log P + \log Y
            \\
            0 &= \log P + 0
            \\
            P &= 1
        \end{split}
    \]
    
    \pagebreak

    \part
    
    Now suppose that the growth rate of the desired level of ouput rises
    such that $\Delta \log Y_t^* = 0.02$ but the money supply remains fixed
    at $\log M_t = 0$ and velocity remains fixed at $\log V=0$. Solve for the
    steady state inflation rate (i.e., steady state value of $\Delta \log P_t$)
    and the steady state output gap (i.e., the steady state value of $\log Y_t
    - \log Y_t^*$). \textit{Hint}: You should guess that $\Delta \log P_t$ is 
    constant in this steady state (i.e., the change in the log price level is
    constant). Then, subtract the time $t - 1$ version of the PS equation above 
    from the time $t$ version of this equation. The resulting equation should 
    help you solve the problem.
    \\ \\
    \solution

    I will first find the steady state increase in output by rewriting the left-
    hand side of the price setting equation for a change in the price level 
    $\Delta \log P_t$. I'll then subtract from it a $t-1$ indexed version of the
    equation and assume that $\Delta \log P_t$ is constant in steady state.
    \[
        \begin{split}
            \log P_{t+1} - \log P_t &= \theta (\log Y_t - \log Y_t^*)
            \\
            \Delta \log P_t &= \theta (\log Y_t - \log Y_t^*)
            \\
            \Delta \log P_t - \Delta \log P_{t-1} &= \theta (\log Y_t - \log Y_t^*) - \theta (\log Y_{t-1} - \log Y_{t-1}^*)
            \\
            \Delta \log P - \Delta \log P &= \theta (\log Y_t - \log Y_{t-1} - (\log Y_t^* - \log Y_{t-1}^*))
            \\
            0 &= \theta (\Delta \log Y_t - \Delta \log Y_t^*)
            \\
            \Delta \log Y_t &= \Delta \log Y_t^*
        \end{split}
    \]

    To find the steady state value of $\Delta \log P_t$, I will now use the money
    market equilibrium condition and substitute my value for $\Delta \log Y_t$ into
    a difference between $t$ and $t-1$ indexed versions of the equation where $\log M_t$
    and $\log V$ are both equal to zero (the money supply and velocit remain fixed).
    \[
        \begin{split}
            \log M_t + \log V &= \log P_t + \log Y_t
            \\
            0 + 0 &= \log P_t + \log Y_t
            \\
            0 &= \log P_t - \log P_{t-1} + \log Y_t - \log Y_{t-1}
            \\
            0 &= \Delta \log P_t + \Delta \log Y_t
            \\
            \Delta \log P_t &= - \Delta \log Y_t
            \\
            \Delta \log P_t &= - \Delta \log Y_t^*
        \end{split}
    \]

    So the price level is decreasing over time proportionally with the increase in
    the desired level of output, \fbox{$\Delta \log P_t = -0.02$}. To find the steady 
    state output gap, I will substitute this $\Delta \log P_t$ into the price setting 
    equation.
    \[
        \begin{split}
            \log P_{t+1} - \log P_t &= \theta (\log Y_t - \log Y_t^*)
            \\
            \Delta \log P_t &= \theta (\log Y_t - \log Y_t^*)
            \\
            \frac{\Delta \log P_t}{\theta} &= \log Y_t - \log Y_t^*
            \\
            - \frac{\Delta \log Y_t^*}{\theta} &= \log Y_t - \log Y_t^*
        \end{split}
    \]
    
    This shows that the steady state output gap is negative and in proportion to
    the increase in the desired level of output, \fbox{$\log Y_t - \log Y_t^* = 
    - \frac{0.02}{\theta}$}. Output takes time to catch up to the ever changing 
    desired level of output, so the output gap is negative.

    \pagebreak

    \part

    Starting from the steady state you calculated in part A), suppose that
    the growth rate of the desired level of output rises such that $\Delta \log 
    Y_t^* = 0.02$. Solve for the evolution of $\log P_t$ and $\log Y_t$ over the
    next 20 periods. Assume that $\theta = 0.5$. Again assume that the 
    money supply remains fixed at $\log M_t = 0$ and velocity remains fixed at
    $\log V = 0$. Graph the result. 
    \\ \\
    \solution

    \begin{center}
    \begin{tikzpicture}
        \begin{axis}[
            width=0.9\textwidth, % Make the plot wider
            height=0.5\textwidth,
            xmin=-1,
            xmax=20,
            axis line style={line width=1pt},
            xlabel={Time $t$},
            ylabel={},
            title style={font=\large, align=center},
            xlabel style={font=\large},
            legend style={
                at={(0.5,1.15)}, % Place legend below the plot
                anchor=north,
                legend columns=-1, % Arrange legend in a single row
                font=\small,
                draw=black % Add a border to the legend
            },
            grid=both, % Add gridlines
            minor grid style={dashed, gray!30}, % Style for minor gridlines
            major grid style={solid, gray!60}, % Style for major gridlines
            ]
            \draw[thick, black] (axis cs: 0, \pgfkeysvalueof{/pgfplots/ymin}) -- 
                                (axis cs: 0, \pgfkeysvalueof{/pgfplots/ymax});
            \draw[thick, black] (axis cs: \pgfkeysvalueof{/pgfplots/xmin}, 0) -- 
                                (axis cs: \pgfkeysvalueof{/pgfplots/xmax}, 0);
            
            \addplot[blue, thick, mark=*] 
            table[x=T, y=P, col sep=comma] {deflation.csv};
            \addlegendentry{$\log P_t$}
    
            \addplot[red, thick, mark=square*] 
            table[x=T, y=O, col sep=comma] {deflation.csv};
            \addlegendentry{$\log Y_t$}
        \end{axis}
    \end{tikzpicture}
    \end{center}
    
    
\end{homeworkProblem}