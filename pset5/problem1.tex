\begin{homeworkProblem}[1]
    In this problem, we propose to explore two famous episodes when the price 
    level fell substantially. One was in England between 1250 and 1490. The 
    other episode was in the United States between 1879 and 1896.
    \\ \\
    One commonly argued reason why the price level fell durin gtheese periods 
    was that the economies in question were growing and this was putting downward
    pressure on prices as the supply of money was not growing fast enough to
    offset the growth in the economy. In this problem, we use an extension of
    the medieval economy model to formalize this argument.
    \\ \\
    Consider a version of the medieval economy model in which the desired level
    of output may change over time. Since it may change over time, we denote it
    as $Y_t^*$ rather than $Y^*$. You can think of the reasons why the desired 
    level of output can change over time being either that the size of the 
    population changes over time or that the level of technology changes over 
    time so each person can produce more output per hour worked.
    \\ \\
    Since the desired level of output can change, the price setting equation
    becomes:
    
    \begin{flalign*}
        & \quad \quad \text{PS:} \quad \quad \log P_{t+1} - \log P_t = \theta (\log Y_t - \log Y_t^*) &\\
    \end{flalign*}
    
    Suppose the money market equilibrium condition is the same as before:
    
    \begin{flalign*}
        & \quad \quad \text{MME:} \quad \quad \log M_t + \log V = \log P_t + \log Y_t &\\
    \end{flalign*}
    
    A) Suppose for simplicity that $\log Y_t^*$ is a constant equal to 0. Solve
    for the steady state level of output and steady state price level when $\log M_t 
    = 0$ and $\log V=0$. 
    \\ \\
    B) Now suppose that the growth rate of the desired level of ouput rises
    such that $\Delta \log Y_t^* = 0.02$ but the money supply remains fixed
    at $\log M_t = 0$ and velocity remains fixed at $\log V=0$. Solve for the
    steady state inflation rate (i.e., steady state value of $\Delta \log P_t$)
    and the steady state output gap (i.e., the steady state value of $\log Y_t
    - \log Y_t^*$). \textit{Hint}: You should guess that $\Delta \log P_t$ is 
    constant in this steady state (i.e., the change in the log price level is
    constant). Then, subtract the time $t - 1$ version of the PS equation above 
    from the time $t$ version of this equation. The resulting equation should 
    help you solve the problem.
    \\ \\
    C) Starting from the steady state you calculated in part A), suppose that
    the growth rate of the desired level of output rises such that $\Delta \log 
    Y_t^* = 0.02$. Solve for the evolution of $\log P_t$ and $\log Y_t$ over the
    next 20 periods. Please assume that $\theta = 0.5$. Again assume that the 
    money supply remains fixed at $\log M_t = 0$ and velocity remains fixed at
    $\log V = 0$. Please graph the result. 
    
    \pagebreak
    
\end{homeworkProblem}