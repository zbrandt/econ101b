\begin{homeworkProblem}[3]
    One view of the cause of the Great Depression is that it is primarily a 
    massive negative money supply shock associated with France (and the U.S.)
    hoarding gold and exchanging sterling for gold. This question asks you to
    analyze the implications of such a shock for output, price, and interest 
    rates through the lens of our business cycle model in its IS-LM-PS 
    incarnation.
    \\ \\
    Consider the following version of our business cycle model:

    \begin{flalign*}
        & \quad \quad \text{LM curve:} \quad \quad \Delta \log M_t - \pi_t = - \phi i_t + \phi i_t-1 + \widetilde{Y}_t - \widetilde{Y}_{t-1} + \Delta \log V_t &\\
        & \quad \quad \text{IS curve:} \quad \quad \widetilde{Y}_t = \overline{a} - \overline{b} (R_t - \overline{r}) &\\
        & \quad \quad \text{Fisher equation:} \quad \quad R_t = i_t - \pi_t &\\
        & \quad \quad \text{Price setting equation:} \quad \quad \pi_t = \theta \widetilde{Y}_{t-1} &\\
    \end{flalign*}
    
    The Fisher equation is written assuming adaptive expectations (i.e.,
    $E_t \pi_{t+1} = \pi_t$. Also, the version of the LM curve written above
    comes from taking a first-difference of the LM curve in class and changing 
    variables as before. Throughout this question, you may assume that $\overline{r}
    0.02$, $\phi = 0.5$, $\overline{a} = 0$, $\overline{b} = 1$, and $\theta = 0.2$.
    \\ \\
    A) Suppose that the economy starts off in a steady state with $\Delta \log M_t
    = 0$, $\Delta \log v_t = 0$. Calculate the steady state value of the output 
    gap, inflation, the real interest rate, and the nominal interest rates.
    \\ \\
    B) Suppose that at time 0, the money supply in this economy falls by 0.2 log 
    points (i.e., $\Delta \log M_0 = -0.2$) and remains at that new lower level 
    going forward (i.e., $\Delta \log M_0 = 0$ for $t > 0$). Plot the evolution
    of output, inflation, the nominal interest rate and the real interest rate 
    over time from period 0 to period 20. (You may want to start your figures at 
    period -5 just to have pre-period in the figures to be able to see the response 
    of the economy better. Also, you may ignore the zero lower bound on nominal 
    interest rates. If you don’t know what that is, this is OK. We will cover it
    in a few weeks.)
    \\ \\
    C) Describe the evolution of the economy in response to the shock in part B)
    graphically using the IS-LM diagram.
    
    \pagebreak
    
\end{homeworkProblem}