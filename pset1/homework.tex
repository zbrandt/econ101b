\documentclass{article}

\usepackage{fancyhdr}
\usepackage{extramarks}
\usepackage{amsmath}
\usepackage{amsthm}
\usepackage{amsfonts}
\usepackage{tikz}
\usepackage[plain]{algorithm}
\usepackage{algpseudocode}
\usepackage{pgfplots}

\usetikzlibrary{automata,positioning}

%
% Basic Document Settings
%

\topmargin=-0.45in
\evensidemargin=0in
\oddsidemargin=0in
\textwidth=6.5in
\textheight=9.0in
\headsep=0.25in

\linespread{1.1}

\pagestyle{fancy}
\lhead{\hmwkAuthorName}
\chead{}
\rhead{\hmwkClass\ (\hmwkClassInstructor\ \hmwkClassTime): \hmwkTitle}
\lfoot{\lastxmark}
\cfoot{\thepage}

\renewcommand\headrulewidth{0.4pt}
\renewcommand\footrulewidth{0.4pt}

\setlength\parindent{0pt}

\pgfplotsset{compat=1.18}

%
% Create Problem Sections
%

\newcommand{\enterProblemHeader}[1]{
    \nobreak\extramarks{}{Problem \arabic{#1} continued on next page\ldots}\nobreak{}
    \nobreak\extramarks{Problem \arabic{#1} (continued)}{Problem continued on next page\ldots}\nobreak{}
}

\newcommand{\exitProblemHeader}[1]{
    \nobreak\extramarks{Problem \arabic{#1} (continued)}{Problem \arabic{#1} continued on next page\ldots}\nobreak{}
    \stepcounter{#1}
    \nobreak\extramarks{Problem \arabic{#1}}{}\nobreak{}
}

\setcounter{secnumdepth}{0}
\newcounter{partCounter}
\newcounter{homeworkProblemCounter}
\setcounter{homeworkProblemCounter}{1}
\nobreak\extramarks{Problem \arabic{homeworkProblemCounter}}{}\nobreak{}

%
% Homework Problem Environment
%
% This environment takes an optional argument. When given, it will adjust the
% problem counter. This is useful for when the problems given for your
% assignment aren't sequential. See the last 3 problems of this template for an
% example.
%
\newenvironment{homeworkProblem}[1][-1]{
    \ifnum#1>0
        \setcounter{homeworkProblemCounter}{#1}
    \fi
    \section{Problem }
    \setcounter{partCounter}{1}
    \enterProblemHeader{homeworkProblemCounter}
}{
    \exitProblemHeader{homeworkProblemCounter}
}

%
% Homework Details
%   - Title
%   - Due date
%   - Class
%   - Section/Time
%   - Instructor
%   - Author
%

\newcommand{\hmwkTitle}{Problem Set\ \#1}
\newcommand{\hmwkDueDate}{September 22, 2024}
\newcommand{\hmwkClass}{Macroeconomics}
\newcommand{\hmwkClassTime}{Section 102}
\newcommand{\hmwkClassInstructor}{Prof. Barnichon}
\newcommand{\hmwkAuthorName}{\textbf{Zachary Brandt}}

%
% Title Page
%

\title{
    \vspace{2in}
    \textmd{\textbf{\hmwkClass:\ \hmwkTitle}}\\
    \normalsize\vspace{0.1in}\small{Due\ on\ \hmwkDueDate\ at 10:00pm}\\
    \vspace{0.1in}\large{\textit{\hmwkClassInstructor\ \hmwkClassTime}}
    \vspace{3in}
}

\author{\hmwkAuthorName}
\date{}

\renewcommand{\part}[1]{\textbf{\large Part \Alph{partCounter}}\stepcounter{partCounter}\\}

%
% Various Helper Commands
%

% Useful for algorithms
\newcommand{\alg}[1]{\textsc{\bfseries \footnotesize #1}}

% For derivatives
\newcommand{\deriv}[1]{\frac{\mathrm{d}}{\mathrm{d}x} (#1)}

% For partial derivatives
\newcommand{\pderiv}[2]{\frac{\partial}{\partial #1} (#2)}

% Integral dx
\newcommand{\dx}{\mathrm{d}x}

% Alias for the Solution section header
\newcommand{\solution}{\textbf{\large Solution}}

% Probability commands: Expectation, Variance, Covariance, Bias
\newcommand{\E}{\mathrm{E}}
\newcommand{\Var}{\mathrm{Var}}
\newcommand{\Cov}{\mathrm{Cov}}
\newcommand{\Bias}{\mathrm{Bias}}

\begin{document}

\pgfplotsset{
    standard/.style={
        axis line style = thick,
        trig format=rad,
        enlargelimits,
        axis x line=middle,
        axis y line=middle,
        enlarge x limits=0.15,
        enlarge y limits=0.15,
        every axis x label/.style={at={(current axis.right of origin)}, anchor=north west},
        every axis y label/.style={at={(current axis.above origin)},anchor=south east},
        % grid=both,
        ticklabel style={font=\large, fill=white}
    }
}

\maketitle

\pagebreak

\begin{homeworkProblem}
    Politicians seeking to garner support for lower taxes will sometimes argue that lowering taxes will
    actually lead to an increase in tax revenue because this will increase output sharply. Clearly, if tax
    rates were 100\%, little revenue would be raised since few people would work much. So, at sufficiently
    high tax rates, this argument is valid. But how high do taxes need to be for this to be the case?
    \\ \\
    The argument that lowering taxes will raise revenue was famously made by economist Arthur Laffer
    in a dinner meeting in 1974 with – then – Ford Administration officials Dick Cheney and Donald
    Rumsfeld. Laffer is said to have illustrated his argument by drawing a curve on his napkin showing
    how tax revenue rises with the tax rate at low tax rates but eventually falls back down to zero at 100\%
    tax rates. A graph of tax revenue as a function of tax rates has since been called the Laffer curve.
    Having a tax rate higher than the tax rate that maximizes revenue (i.e., a tax rate on the downward
    sloping portion of the curve) is called being on the wrong side of the Laffer curve.
    \\ \\
    In this question, you will solve for the “top of the Laffer curve,” i.e., the tax rate that maximizes tax
    revenue, in a simple model.
    \\

    \pagebreak

    \part

    Suppose output in the economy is produced with only labor (no capital, no land, etc., for
    simplicity). Suppose the production function is $Y= AL$, where $Y$ denotes total output in the economy,
    $A$ denotes productivity, and $L$ denotes total labor in the economy. Suppose firms take wages $w$ as
    given. Derive the labor demand curve in this economy. Plot the labor demand curve in ($w$, $L$) space
    (i.e., with $w$ on the y-axis and $L$ on the x-axis). In one or two sentences, comment on why it makes
    sense that the labor demand curve takes this form in this model.
    \\

    \solution
    \\
    \begin{figure}[h]
        \centering
        \begin{tikzpicture}
            \begin{axis}[standard, 
                        xtick={1,2,3,4,5},
                        ytick={1,2,3,4,5},
                        xticklabels={},
                        yticklabels={,,$A$,,},
                        xlabel={Labor $L$},
                        ylabel={Wage $w$},
                        samples=1000,
                        xmin=0,
                        xmax=5,
                        ymin=0,
                        ymax=5,
                        % extra y ticks={3},         
                        % extra y tick style={fill=white, font=\large}
                        ]
                % \node[anchor=center,label=south west:$O$] at (axis cs:0,0){};                
                \addplot[line width=0.8mm,red,domain={0:5.5}]{3};
            \end{axis}
        \end{tikzpicture}
    \end{figure}
    
    In a competitive market, firms hire until the marginal product of labor is equal to the wage $w$. This behavior determines the labor demand curve. 
    The derivative of the production function quantifies the marginal product of labor: the marginal change in output with respect to a change in labor.
     

    \[
        w = \frac{\partial Y}{\partial L} = \frac{\partial (AL)}{\partial L} =  A        
    \]

    This shows that the wage is equal to the productivity of labor $A$, which is constant in this economy.
    The labor demand curve therefore does not depend on $L$, meaning there are no diminishing returns to labor. 
    This explains why the labor demand curve takes on this horizontal form in this model.

    \pagebreak

    \part
    
    Suppose each household’s utility function is

    \[
        U = \log(C) - \psi \frac{H^{1+\frac{1}{\eta}}}{1+\eta^{-1}},
    \]

    where $C$ denotes per capita consumption, $H$ denotes per capita hours, and $\eta$ and $\psi$ are parameters.
    Suppose that all households are identical. This implies that they will all consume the same amount in equilibrium and 
    supply the same numbers of hours of labor. Each household's budget constraint is

    \[
        C = (1-\tau_{l})wH + T,
    \]

    where $\tau_{l}$ denotes the labor income tax in this economy and $T$ denotes a lump sum transfer from the
    government to the household. For simplicity, we assume that the government redistributes all tax
    receipts back to the households lump sum. ``Lump sum" means that the household takes the amount
    of transfers it receives as given—i.e., it believes that it can't affect these with its actions. Derive the
    household's labor supply curve. 
    \\ \\ 
    \solution

    The household maximizes utility subject to the budget constraint. Since the household's consumption is a function of hours,
    we can eliminate $C$ from the utility function by substitution. 
    
    \[
        U(C) - V(H) = \log((1-\tau_{l})wH + T) - \psi \frac{H^{1+\frac{1}{\eta}}}{1+\eta^{-1}}
    \]

    Now we can maximize the utility function with respect to one variable $H$. To do this, we differentiate the utility function
    with respect to $H$ and solve for when the result equals zero. This way we can find an equation that implicitly desribes 
    how much labor $H$ the household will supply as a function of the wage $w$, assuming all other variables are constant.

    \[
        \begin{split}
            \frac{\partial U}{\partial H} &= \frac{(1-\tau_{l})w}{(1-\tau_{l})wH + T} - \psi H^{\frac{1}{\eta}} 
            \\
            \text{Set equal to zero} \;\;\; 0 &= \frac{(1-\tau_{l})w}{(1-\tau_{l})wH + T} - \psi H^{\frac{1}{\eta}} 
            \\
            \psi H^{\frac{1}{\eta}} &= \frac{(1-\tau_{l})w}{(1-\tau_{l})wH + T}
            \\
            \psi H^{\frac{1}{\eta}} &= \frac{(1-\tau_{l})}{(1-\tau_{l})wH + T} \cdot w
        \end{split}
    \]
    \\
    The equation above describes what is true for households optimally trading off the utility of consumption, 
    and the disulity of work. On the left hand side is the marginal disutility of work in utils per hour, $V'(H)$. 
    On the right hand side is the marginal value of extra consumption afforded by working an extra hour. 
    This is equal to the wage $w$ times the marginal value of consumption, $U'(C)$, resulting
    in utils per hour worked again. 
    \\

    The equation therefore shows that for households to optimize they must set the marginal benefit of working an
    extra hour to its marginal cost. This is the labor supply curve. It describes how much labor in hours worked
    $H$ a household will supply as a function of the wage $w$.

    \pagebreak

    \part

    Suppose that there are $N$ households in the economy. Total tax revenue will then be 
    $\tau_{l}wHN$, i.e., the tax rate times household labor income ($wH$) times the number of households.
    Suppose that each household recieves an equal share of this tax revenue as a lump sum transfer from
    the government. Use this fact, the household's budget constraint, its labor supply curve, and the labor
    demand equation to show that hours worked per person in this economy can be expressed as

    \[
        H = (1-\tau_{l})^{\frac{\eta}{\eta+1}} \psi^{\frac{-\eta}{\eta + 1}}.
    \]

    Plot this curve in ($\tau_{l}$, $H$) space for $\eta = 1$ and $\psi = 2$. In one or two 
    sentences, describe the intuition for the shape of this curve. 
    \\ \\

    \solution

    We can express the lump sum transfers $T$ from the government to the households with the total 
    tax revenue. If each household recieves an equal share of the tax revenue, then the lump sum transfer
    is simply this total tax revenue divided by the number of households.

    \[
        T = \frac{\tau_{l}wHN}{N} = \tau_{l}wH
    \]
    
    We can substitute this expression for $T$ into the labor supply curve equation from part (b) to find
    the hours worked per person in this economy as a function of the tax rate $\tau_l$ and parameters 
    $\eta$ and $\psi$.
    \\

    \begin{minipage}{0.5\textwidth}
        \[
            \begin{split}
                \psi H^{\frac{1}{\eta}} &= \frac{(1-\tau_{l})w}{(1-\tau_{l})wH + \tau_{l}wH}
                \\
                &= \frac{(1-\tau_l)}{(1-\tau_l)H+\tau_lH}
                \\
                &= \frac{1}{H} \cdot \frac{1-\tau_l}{(1+\tau_l)+\tau_l}
                \\
                &= \frac{1}{H} \cdot \frac{1-\tau_l}{1}
                \\
                H \cdot \psi H^{\frac{1}{\eta}} &= 1-\tau_l
                \\
                H^{\frac{\eta}{\eta}} \cdot \psi H^{\frac{1}{\eta}} &= 
                \\
                \psi H^{\frac{\eta+1}{\eta}} &= 
                \\
                \psi^{\frac{\eta}{\eta+1}} H &= (1-\tau_l)^{\frac{\eta}{\eta + 1}}
                \\
                H &= (1-\tau_{l})^{\frac{\eta}{\eta+1}} \psi^{\frac{-\eta}{\eta + 1}}
            \end{split}
        \]
    \end{minipage}
    \begin{minipage}{0.5\textwidth}
        \centering
        \begin{tikzpicture}
            \begin{axis}[standard, 
                        ytick=\empty,
                        xtick={1},
                        xticklabels={1},
                        yticklabels={},
                        xlabel={Tax $\tau_l$},
                        ylabel={Hours $H$},
                        samples=10000, 
                        xmin=0,
                        xmax=1,
                        ymin=0,
                        ymax=1]
                \node[anchor=center,label=south east:{\large 0}] at (axis cs:0,0){};                
                \addplot[line width=0.8mm,red,domain={0:5}]{(1-x)^(0.5) * 2^(-0.5)};
            \end{axis}
        \end{tikzpicture}
    \end{minipage}
    \\ \\ \\ 
    The curve is downward sloping because as the tax rate increases from 0 to 1, the amount of hours worked per person decreases. 
    The intuition is that as the tax rate increases, the wage that households receive decreases for each hour worked, 
    which incentivizes them to work less.

    \pagebreak

    \part

    Recall that tax revenue is $\tau_{l}wHN$. The Laffer curve is a tax revenue as a function of the
    tax rate and only exogenous variables and parameters. Use the expression you derived in part (c) as
    well as the labor demand curve to derive an expression for tax revenue that is a function of only the
    tax rate $\tau_{l}$ and the exogenous variables ($A$, $N$) and parameters ($\eta$, $\psi$). 
    Plot this function assuming that $\eta=1$, $\psi=2$, $A=1$, and $N=1$. 
    \\

    \solution

    Using our expression for hours worked per person from part (c) and the labor demand curve in part (a),
    we can substitute these into $\tau_{l}wHN$ to find an expression for tax revenue as a function of $\tau_l$
    and the exogenous variables and parameters.

    \[
        \begin{split}
            \tau_{l}wHN &= \tau_{l}w (1-\tau_{l})^{\frac{\eta}{\eta+1}} \psi^{\frac{-\eta}{\eta + 1}} N \;\;\; \text{substitute from part (c)}
            \\
            &= \tau_{l}A (1-\tau_{l})^{\frac{\eta}{\eta+1}} \psi^{\frac{-\eta}{\eta + 1}} N
        \end{split}
    \]
    
    \begin{figure}[h]
        \centering
        \begin{tikzpicture}
            \begin{axis}[standard,
                        ytick=\empty, 
                        xtick={1},
                        xticklabels={1},
                        yticklabels={},
                        xlabel={Tax rate $\tau_l$},
                        ylabel={Tax revenue},
                        samples=10000, 
                        xmin=0,
                        xmax=1,
                        ymin=0,
                        ymax=1]
                \node[anchor=center,label=south east:{\large 0}] at (axis cs:0,0){};                
                \addplot[line width=0.8mm,red,domain={0:5}]{x * (1-x)^(0.5) * 2^(-0.5)};
            \end{axis}
        \end{tikzpicture}
    \end{figure}
    
    This plot demonstrates features of a Laffer curve described in the problem description.
    From the plot, we can see that tax revenue is minimized at a tax rate of 0 and 1, and that tax revenues 
    rise going up from 0, reach a maximum, and then fall back down again as part of being on the wrong side 
    of the Laffer curve. 

    \pagebreak

    \part

    Find the top of the Laffer curve. In other words, derive an expression for the tax rate
    that yields maximum tax revenue as a function of the exogenous parameters ($\eta$, $\psi$).
    \\

    \solution 

    To find the tax rate that maximizes tax revenue, we can take the derivative of the tax revenue function with respect to the tax rate $\tau_l$ and set it equal to zero.

    \[
        \begin{split}
            \frac{\partial (\text{Tax revenue})}{\partial \tau_l} &= \frac{\partial (\tau_{l}A(1-\tau_{l})^{\frac{\eta}{\eta + 1}}\psi^{\frac{-\eta}{\eta + 1}}N )}{\partial \tau_l}
            \\
            &= A (1-\tau_{l})^{\frac{\eta}{\eta+1}} \psi^{\frac{-\eta}{\eta + 1}} N - A \tau_l \frac{\eta}{\eta+1} (1-\tau_{l})^{\frac{-1}{\eta+1}} \psi^{\frac{-\eta}{\eta + 1}} N
            \\
            \text{Set equal to zero} \;\;\;\;\;\; 0 &= A (1-\tau_{l})^{\frac{\eta}{\eta+1}} \psi^{\frac{-\eta}{\eta + 1}} N - A \tau_l \frac{\eta}{\eta+1} (1-\tau_{l})^{\frac{-1}{\eta+1}} \psi^{\frac{-\eta}{\eta + 1}} N
            \\
            &= (1-\tau_{l})^{\frac{\eta}{\eta+1}} - \tau_l \frac{\eta}{\eta+1} (1-\tau_{l})^{\frac{-1}{\eta+1}}
            \\
            \tau_l \frac{\eta}{\eta+1} (1-\tau_{l})^{\frac{-1}{\eta+1}} &= (1-\tau_{l})^{\frac{\eta}{\eta+1}}
            \\
            \tau_l \frac{\eta}{\eta+1} &= (1-\tau_{l})
            \\
            \eta \tau_l &= (1-\tau_{l})(\eta+1)
            \\
            \eta \tau_l &= \eta + 1 - \eta \tau_l - \tau_l
            \\
            2\eta \tau_l + \tau_l &= \eta + 1
            \\
            \tau_l &= \frac{\eta + 1}{2\eta + 1}
        \end{split}
    \]

    The tax rate that maximizes tax revenue is $\frac{\eta + 1}{2\eta + 1}$, and solves for the top of the Laffer curve.
    It is only a function of the parameter $\eta$. Using our specification from part (d), the tax rate that maximizes 
    tax revenue is $\frac{1 + 1}{2(1) + 1} = \frac{2}{3}$. Taking a look at our plot of tax revenue where 
    $\eta = 1$, $\psi = 2$, $A = 1$, and $N = 1$, we can see that this is the case.
    \\ \\
    
    \begin{figure}[h]
        \centering
        \begin{tikzpicture}
            \begin{axis}[standard,
                        ytick=\empty, 
                        xtick={1},
                        xticklabels={1},
                        yticklabels={},
                        xlabel={Tax rate $\tau_l$},
                        ylabel={Tax revenue},
                        samples=10000, 
                        xmin=0,
                        xmax=1,
                        ymin=0,
                        ymax=1]
                \node[anchor=center,label=south east:{\large 0}] at (axis cs:0,0){};
                \node[anchor=center,label=south:{\small $0.\overline{666}$ }] at (axis cs:2/3,0){};
                % Vertical asymptote at x = 2/3
                \addplot[dashed, gray, domain=0:1] coordinates {(2/3,0) (2/3,1)};
            
                % Horizontal asymptote at y = 0.272
                \addplot[dashed, gray, domain=0:1] {0.28};
                
                \addplot[line width=0.8mm,red,domain={0:5}]{x * (1-x)^(0.5) * 2^(-0.5)};
            \end{axis}
        \end{tikzpicture}
    \end{figure}

    \pagebreak

    \part

    There is a lively debate among economists about what an appropriate value for the
    parameter $\eta$ is. This parameter is called the Frisch elasticity of labor supply. It is the percentage
    change in hours worked when wages change by 1\% but holding fixed consumption. Based on data
    for middle aged males, many labor economists believe that $\eta$ is quite small, e.g., $\eta \approx 0.5$.
    Macroeconomists, however, often point to changes in participation decisions as evidence for high
    labor supply elasticities, e.g., early retirement, unemployment, etc. Macroeconomists therefore
    sometimes use values for $\eta$ that are quite above 1, e.g., $\eta \approx 3$. Calculate the tax rate that yields
    maximal tax revenue (i.e., the top of the Laffer curve) for the following three values of $\eta$: 0.5, 1, 3.
    \\

    \solution 

    We can use the formula we derived in part (e) to calculate the tax rate that maximizes tax revenue for the three values of $\eta$.

    \[
        \begin{split}
            \eta = 0.5: \;\;\;\;\;\; \tau_l &= \frac{0.5 + 1}{2(0.5) + 1} = \frac{1.5}{2} = 0.75
            \\
            \eta = 1: \;\;\;\;\;\; \tau_l &= \frac{1 + 1}{2(1) + 1} = \frac{2}{3} = 0.\overline{666}
            \\
            \eta = 3: \;\;\;\;\;\; \tau_l &= \frac{3 + 1}{2(3) + 1} = \frac{4}{7} \approx 0.571
        \end{split}
    \]

    From these three examples, higher values of $\eta$ lead to lower tax rates that maximize tax revenue. 
    This could be because higher values of $\eta$ imply that households are more sensitive to changes in wages, 
    and therefore will respond with working less as the tax rate increases.

\end{homeworkProblem}

\pagebreak


\end{document}
