\begin{homeworkProblem}[2]
    Consider the business cycle model with the monetary policy rule we saw in
    class (see also chapter 13 of the Jones textbook):
    
    \begin{flalign*}
        & \quad \quad \text{Monetary policy rule:} \quad R_t - \overline{r} = \overline{m} (\pi_t - \overline{\pi})&\\
        & \quad \quad \text{IS curve:} \quad \widetilde{Y}_t = \overline{a}_t - \overline{b} (R_t - \overline{r})&\\
        & \quad \quad \text{Phillips} \quad \pi_t = \pi_{t-1} + \overline{v} \widetilde{Y}_t + \overline{o}_t&\\
    \end{flalign*}
    
    Here we allow the aggregate demand and cost-push shocks to vary with time (so,
    they have time subscripts).
    \\ \\
    A) Combine the monetary policy rule and the IS curve to derive an ``aggregate
    demand'' equation for this model.
    \\ \\
    B) This leaves two equations and two unknown endogenous variables (output gap and
    inflation). Solve these two equations, i.e., use these two equations to express the
    endogenous variables as functions of only past endogenous variables, exogenous variables,
    and parameters. 
    \\ \\
    C) Use the result from part B) and the original monetary policy rule to express the real
    interest rate as a function of only past endogenous variables, exogenous variables, and
    parameters. 
    \\ \\
    Suppose $\overline{r} = 3\%$, $\overline{\pi} = 2\%$, $\overline{b}=1/3$, $\overline{m} =
    1$, $\overline{v}=1$. Suppose furthermore that $\pi_{-1} = 2\%$. Suppose $\overline{a}_t$
    and $\overline{o}_t$ are zero at all times unless otherwise noted. 
    \\ \\
    D) Calculate the values of inflation, the output gap, and real interest rate in period 0.
    \\ \\
    E) Now suppose the economy is hit by a positive aggregate demand shock that lasts for 3 
    periods. More specifically, suppose $\overline{a}_1 = \overline{a}_2 = \overline{a}_3 = 2\%$.
    After period 3, the aggregate demand shock returns to 0. Using the three equation model,
    calculate the evolution of inflation, the output gap, and the real interest rate form period
    1 to period 10. Plot the resulting time series for these three variables. You may find it useful
    to do this in Excel.
    \\ \\
    F) Starting form the point calculated for period 0 in part D), suppose the economy is instead
    hit by a positive aggregate supply shock that lasts for 3 periods. More specifically, suppose
    $\overline{o}_1 = \overline{o}_2 = \overline{o}_3 = 2\%$. After period 3, the aggregate supply
    shock returns to 0. Using the three equation model, calculate the evolution of inflation, the
    output gap, and the real interest rate from period 1 to period 10. Plot the time series for 
    these three variables. 
    \\ \\
    G) Comment on the difference between the response of the economy to an aggregate demand schock
    and an aggregate supply shock.
    
    \pagebreak
    
\end{homeworkProblem}