\begin{homeworkProblem}[2]
    Consider the business cycle model with the monetary policy rule we saw in
    class (see also chapter 13 of the Jones textbook):
    
    \begin{flalign*}
        & \quad \quad \text{Monetary policy rule:} \quad R_t - \overline{r} = \overline{m} (\pi_t - \overline{\pi})&\\
        & \quad \quad \text{IS curve:} \quad \widetilde{Y}_t = \overline{a}_t - \overline{b} (R_t - \overline{r})&\\
        & \quad \quad \text{Phillips} \quad \pi_t = \pi_{t-1} + \overline{v} \widetilde{Y}_t + \overline{o}_t&\\
    \end{flalign*}
    
    Here we allow the aggregate demand and cost-push shocks to vary with time (so,
    they have time subscripts).
    \\ \\
    A) Combine the monetary policy rule and the IS curve to derive an ``aggregate
    demand'' equation for this model.
    \\ \\
    B) This leaves two equations and two unknown endogenous variables (output gap 
    and inflation). Solve these two equations, i.e., use these two equations to 
    express the endogenous variables as functions of only past endogenous variables, 
    exogenous variables, and parameters. 
    \\ \\
    C) Use the result from part B) and the original monetary policy rule to express
    the real interest rate as a function of only past endogenous variables, exogenous
    variables, and parameters. 
    \\ \\
    Suppose $\overline{r} = 3\%$, $\overline{\pi} = 2\%$, $\overline{b}=1/3$, 
    $\overline{m} = 1$, $\overline{v}=1$. Suppose furthermore that $\pi_{-1} = 2\%$.
    Suppose $\overline{a}_t$ and $\overline{o}_t$ are zero at all times unless otherwise
    noted. 
    \\ \\
    D) Calculate the values of inflation, the output gap, and real interest rate in
    period 0.
    \\ \\
    E) Now suppose the economy is hit by a positive aggregate demand shock that lasts
    for 3 periods. More specifically, suppose $\overline{a}_1 = \overline{a}_2 = 
    \overline{a}_3 = 2\%$. After period 3, the aggregate demand shock returns to 0. 
    Using the three equation model, calculate the evolution of inflation, the output gap,
    and the real interest rate form period 1 to period 10. Plot the resulting time series
    for these three variables. You may find it useful to do this in Excel.
    \\ \\
    F) Starting form the point calculated for period 0 in part D), suppose the economy is 
    instead hit by a positive aggregate supply shock that lasts for 3 periods. More 
    specifically, suppose $\overline{o}_1 = \overline{o}_2 = \overline{o}_3 = 2\%$. After
    period 3, the aggregate supply shock returns to 0. Using the three equation model, 
    calculate the evolution of inflation, the output gap, and the real interest rate from
    period 1 to period 10. Plot the time series for these three variables. 
    \\ \\
    G) Comment on the difference between the response of the economy to an aggregate demand 
    schock and an aggregate supply shock.
    
    \pagebreak
    \part 

    Combine the monetary policy rule and the IS curve to derive an ``aggregate demand'' 
    equation for this model.
    \\ \\
    \solution 

    To find the aggregate demand equation for this model I will substitute in the monetary
    policy rule equation for $R_t$ in the IS curve equation to obtain an equation for
    the output gap in terms of inflation
    \[
        \begin{split}
                \widetilde{Y}_t &= \overline{a}_t - \overline{b} (R_t - \overline{r})
                \\
                \widetilde{Y}_t &= \overline{a}_t - \overline{b} (\overline{m} (\pi_t - \overline{\pi}) + \overline{r} - \overline{r})
                \\
                \widetilde{Y}_t &= \overline{a}_t - \overline{b} \overline{m} (\pi_t - \overline{\pi})
        \end{split}
    \]

    This curve describes the aggregate demand mechanism. When inflation increases, the
    central bank raises real rates according to the monetary policy rule, resulting in 
    a decline in consumption and investment, thereby decreasing output, and the output
    gap. 

    \pagebreak
    \part

    This leaves two equations and two unknown endogenous variables (output gap 
    and inflation). Solve these two equations, i.e., use these two equations to 
    express the endogenous variables as functions of only past endogenous variables, 
    exogenous variables, and parameters. 
    \\ \\
    \solution

    To solve for the endogenous variable of inflation, I will substitute in the equation
    I derived for the aggregate demand into the Phillips curve equation.
    \[
        \begin{split}
            \pi_t &= \pi_{t-1} + \overline{v} \widetilde{Y}_t + \overline{o}_t
            \\
            \pi_t &= \pi_{t-1} + \overline{v} (\overline{a}_t - \overline{b} \overline{m} (\pi_t - \overline{\pi})) + \overline{o}_t
            \\
            \pi_t &= \pi_{t-1} + \overline{v} \overline{a}_t - \overline{v} \overline{b} \overline{m} \pi_t + \overline{v} \overline{b} \overline{m} \overline{\pi} + \overline{o}_t
            \\
            \pi_t + \overline{v} \overline{b} \overline{m} \pi_t &= \pi_{t-1} + \overline{v} \overline{a}_t + \overline{v} \overline{b} \overline{m} \overline{\pi} + \overline{o}_t
            \\
            \pi_t (1 + \overline{v} \overline{b} \overline{m}) &= \pi_{t-1} + \overline{v} \overline{a}_t + \overline{v} \overline{b} \overline{m} \overline{\pi} + \overline{o}_t
            \\
            \pi_t &= \frac{\pi_{t-1} + \overline{v} \overline{a}_t + \overline{v} \overline{b} \overline{m} \overline{\pi} + \overline{o}_t}{1 + \overline{v} \overline{b} \overline{m}}
        \end{split}
    \]

    Now to solve for the output gap I can substitute in this equation for the inflation
    into the aggregate demand equation I found earlier
    \[
        \begin{split}
            \widetilde{Y}_t &= \overline{a}_t - \overline{b} \overline{m} (\pi_t - \overline{\pi})
            \\
            \widetilde{Y}_t &= \overline{a}_t - \overline{b} \overline{m} (\frac{\pi_{t-1} + \overline{v} \overline{a}_t + \overline{v} \overline{b} \overline{m} \overline{\pi} + \overline{o}_t}{1 + \overline{v} \overline{b} \overline{m}} - \overline{\pi})
        \end{split}
    \]

    \pagebreak
    \part 

    Use the result from part B) and the original monetary policy rule to express
    the real interest rate as a function of only past endogenous variables, exogenous
    variables, and parameters. 
    \\ \\
    \solution

    To solve for the real interest rate as a function of only past endogenous variables,
    exogenous variables, and parameters, I will again substitute in the equation I found
    for inflation but this time into the monetary policy rule equation
    \[
        \begin{split}
            R_t - \overline{r} &= \overline{m} (\pi_t - \overline{\pi})
            \\
            R_t - \overline{r} &= \overline{m} (\frac{\pi_{t-1} + \overline{v} \overline{a}_t + \overline{v} \overline{b} \overline{m} \overline{\pi} + \overline{o}_t}{1 + \overline{v} \overline{b} \overline{m}} - \overline{\pi})
            \\
            R_t &= \overline{r} + \overline{m} (\frac{\pi_{t-1} + \overline{v} \overline{a}_t + \overline{v} \overline{b} \overline{m} \overline{\pi} + \overline{o}_t}{1 + \overline{v} \overline{b} \overline{m}} - \overline{\pi})
        \end{split}
    \]

    \pagebreak
    \part 

    Suppose $\overline{r} = 3\%$, $\overline{\pi} = 2\%$, $\overline{b}=1/3$, 
    $\overline{m} = 1$, $\overline{v}=1$. Suppose furthermore that $\pi_{-1} = 2\%$.
    Suppose $\overline{a}_t$ and $\overline{o}_t$ are zero at all times unless otherwise
    noted. 
    \\ \\
    Calculate the values of inflation, the output gap, and real interest rate in
    period 0.
    \\ \\
    \solution 

    First to solve for inflation at time $t=0$, I will substitute in the given values into 
    the equation I found for inflation in part B)
    \[
        \begin{split}
            \pi_t &= \frac{\pi_{t-1} + \overline{v} \overline{a}_t + \overline{v} \overline{b} \overline{m} \overline{\pi} + \overline{o}_t}{1 + \overline{v} \overline{b} \overline{m}}
            \\
            \pi_0 &= \frac{\pi_{-1} + \overline{v} \overline{a}_0 + \overline{v} \overline{b} \overline{m} \overline{\pi} + \overline{o}_0}{1 + \overline{v} \overline{b} \overline{m}}
            \\
            \pi_0 &= \frac{2\% + 1 \cdot 0 + 1 \cdot \frac{1}{3} \cdot 1 \cdot 2\% + 0}{1 + 1 \cdot \frac{1}{3} \cdot 1}
            \\
            \pi_0 &= (2\% + \frac{2}{3} \%) \cdot \frac{3}{4}
            \\
            \pi_0 &= \frac{3}{2}\% + \frac{1}{2}\% = 2 \%
        \end{split}
    \]
    So inflation in period zero equals \fbox{$\pi_0 = 2\%$}. To find the output gap, I will
    now substitute in this value for $\pi_0$ into the aggregate demand equation
    \[
        \begin{split}
            \widetilde{Y}_t &= \overline{a}_t - \overline{b} \overline{m} (\pi_t - \overline{\pi})
            \\
            \widetilde{Y}_0 &= \overline{a}_0 - \overline{b} \overline{m} (\pi_0 - \overline{\pi})
            \\
            \widetilde{Y}_0 &= 0 - 1/3 \cdot 1 \cdot (2 \% - 2 \%)
            \\
            \widetilde{Y}_0 &= 0 \%            
        \end{split}
    \]

    So output is at potential at time period 0, i.e., \fbox{$\widetilde{Y}_0 = 0 \%$}. To find 
    the real interest rate at time period 0, I will again substitue in the value I found for 
    $\pi_0$ but now into the equation for the monetary policy rule
    \[
        \begin{split}
            R_t - \overline{r} &= \overline{m} (\pi_t - \overline{\pi})
            \\
            R_0 &= \overline{r} + \overline{m} (\pi_0 - \overline{\pi})
            \\
            R_0 &= 3 \% + 1 \cdot (2 \% - 2 \%)
            \\
            R_0 &= 3 \%
        \end{split}
    \]

    So the real interest rate in period 0 equals the natural rate \fbox{$R_0 = 3 \%$}.

    \pagebreak
    \part

    Now suppose the economy is hit by a positive aggregate demand shock that lasts
    for 3 periods. More specifically, suppose $\overline{a}_1 = \overline{a}_2 = 
    \overline{a}_3 = 2\%$. After period 3, the aggregate demand shock returns to 0. 
    Using the three equation model, calculate the evolution of inflation, the output gap,
    and the real interest rate form period 1 to period 10. Plot the resulting time series
    for these three variables.
    \\ \\
    \solution 
    
    \begin{center}
    \begin{tikzpicture}
        \begin{axis}[
            width=0.9\textwidth, % Make the plot wider
            height=0.5\textwidth,
            xmin=0,
            xmax=10,
            axis line style={line width=1pt},
            xlabel={Time $t$},
            ylabel={},
            title style={font=\large, align=center},
            xlabel style={font=\large},
            legend style={
                at={(0.5,1.15)}, % Place legend below the plot
                anchor=north,
                legend columns=-1, % Arrange legend in a single row
                font=\small,
                draw=black % Add a border to the legend
            },
            grid=both, % Add gridlines
            minor grid style={dashed, gray!30}, % Style for minor gridlines
            major grid style={solid, gray!60}, % Style for major gridlines
            ]
            \draw[thick, black] (axis cs: 0, \pgfkeysvalueof{/pgfplots/ymin}) -- 
                                (axis cs: 0, \pgfkeysvalueof{/pgfplots/ymax});
            \draw[thick, black] (axis cs: \pgfkeysvalueof{/pgfplots/xmin}, 0) -- 
                                (axis cs: \pgfkeysvalueof{/pgfplots/xmax}, 0);
            
            \addplot[blue, thick, mark=*] 
            table[x=t, y=pi, col sep=comma] {demand.csv};
            \addlegendentry{$\pi_t$}

            \addplot[red, thick, mark=*] 
            table[x=t, y=Y, col sep=comma] {demand.csv};
            \addlegendentry{$\widetilde{Y}_t$}
            
            \addplot[brown, thick, mark=*] 
            table[x=t, y=R, col sep=comma] {demand.csv};
            \addlegendentry{$R_t$}

        \end{axis}
    \end{tikzpicture}
    \end{center}

    \pagebreak
    \part 

    Starting form the point calculated for period 0 in part D), suppose the economy is 
    instead hit by a positive aggregate supply shock that lasts for 3 periods. More 
    specifically, suppose $\overline{o}_1 = \overline{o}_2 = \overline{o}_3 = 2\%$. After
    period 3, the aggregate supply shock returns to 0. Using the three equation model, 
    calculate the evolution of inflation, the output gap, and the real interest rate from
    period 1 to period 10. Plot the time series for these three variables. 
    \\ \\
    \solution 

    \begin{center}
    \begin{tikzpicture}
        \begin{axis}[
            width=0.9\textwidth, % Make the plot wider
            height=0.5\textwidth,
            xmin=0,
            xmax=10,
            axis line style={line width=1pt},
            xlabel={Time $t$},
            ylabel={},
            title style={font=\large, align=center},
            xlabel style={font=\large},
            legend style={
                at={(0.5,1.15)}, % Place legend below the plot
                anchor=north,
                legend columns=-1, % Arrange legend in a single row
                font=\small,
                draw=black % Add a border to the legend
            },
            grid=both, % Add gridlines
            minor grid style={dashed, gray!30}, % Style for minor gridlines
            major grid style={solid, gray!60}, % Style for major gridlines
            ]
            \draw[thick, black] (axis cs: 0, \pgfkeysvalueof{/pgfplots/ymin}) -- 
                                (axis cs: 0, \pgfkeysvalueof{/pgfplots/ymax});
            \draw[thick, black] (axis cs: \pgfkeysvalueof{/pgfplots/xmin}, 0) -- 
                                (axis cs: \pgfkeysvalueof{/pgfplots/xmax}, 0);
            
            \addplot[blue, thick, mark=*] 
            table[x=t, y=pi, col sep=comma] {cost.csv};
            \addlegendentry{$\pi_t$}

            \addplot[red, thick, mark=*] 
            table[x=t, y=Y, col sep=comma] {cost.csv};
            \addlegendentry{$\widetilde{Y}_t$}
            
            \addplot[brown, thick, mark=*] 
            table[x=t, y=R, col sep=comma] {cost.csv};
            \addlegendentry{$R_t$}

        \end{axis}
    \end{tikzpicture}
    \end{center}

    \pagebreak
    \part 

    Comment on the difference between the response of the economy to an aggregate demand 
    schock and an aggregate supply shock.
    \\ \\
    \solution 

    The response is the exact same. Both drive up inflation at the same rate, and the response
    from the monetary authority is to raise real rates, thereby decreasing inflation but also
    depressing output below potential for a time. After the shock, inflation, the output gap,
    and the real interest rate gradually approach their steady state values.
    
\end{homeworkProblem}