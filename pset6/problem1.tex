\begin{homeworkProblem}[1]
    In this question, we explore whether government budget deficits are 
    inflationary. In the medieval economy, we assumed that prices adjusted
    to a change in the money supply with a delay. To simplify things in this
    question, we will instead assume that prices are ``perfectly flexible'' ---
    i.e., respond immediately one-for-one to any change in the money supply.
    In other words, we assume that $M_t = P_t$, where $M_t$ is the nominal 
    value of money in the economy in year $t$ --- i.e., the quantity of dollar
    bills outstanding --- and $P_t$ is the price level in year $t$. Notice that
    this is a simplifed version of $M_t \overline{V} = P_t Y_t$, where $\overline{V}
    = 1$, and $Y_t = 1$. 
    \\ \\
    Let's denote the \underline{real value} of government spending on goods and 
    services in year $t$ by $G_t$ and the real value tax revenue in year $t$ by
    $T_t$. Suppose the government can issue one-year debt. Let's denote by $B_t$
    the real value of such debt issued by the government in year $t$ to be paid 
    off with interest in year $t+1$. Finally, let's denote the net real interest
    rate that the government has ot pay on its debt between year $t-1$ and year $t$
    by $R_{t-1}$.
    \\ \\
    The consolidated government budget constraint is then given by
    \[
        \begin{split}
            G_t + (1 + R_{t-1}) B_{t-1} &= T_t + B_t + \frac{M_t - M_{t-1}}{P_t}
        \end{split}
    \]
    The way to interpret this equation is that the terms on the left hand side are
    all the things the government spends money on in year $t$: It purchases goods
    and services, and it pays off the debt it issued in year $t-1$; The terms on the
    right hand side are all the sources of inflow of money into the government 
    coffers in year $t$: tax revenue, income from the sale of new debt, and new money
    issued by the central bank. The equation is written in real terms, which is why
    the change in the quantity of money must be divided by the price level to get how
    much ``stuff'' the new money can buy in year $t$.
    \\ \\
    Suppose that in year $t=0$ the government starts off with no debt to pay off ---
    i.e., $B_{-1} = 0$. Suppose also that the real interest rate that the government
    must pay on its debt is constant at 5\%, i.e., $R_t=0.05$ for all $t$. Furthermore,
    suppose the amount of money in the economy in year $t=-1$ is $M_{-1}$. 
    \\ \\
    Suppose that the institutional setup in this economy is that there is an 
    ``independent'' central bank that chooses $M_t$ in each year and Congress 
    chooses $G_t$ and $T_t$. Jointly, these choices will then determine how much 
    debt will need to be issued by the government in year $t$. 
    \\ \\
    A) Suppose the central bank is determined not to allow any inflation. How 
    should it set the money supply to achieve this?
    \\ \\
    B) Suppose that dysfunctional politics in this country implies that Congress 
    refuses to collect enough taxes to pay for the amount of spending it decides to 
    do. (Alternatively, one could say that it decides to spend more than the amount 
    of taxes it decides to collect. Specifically, suppose $G_t = 0.25$, while $T_t
    = 0.15$. Given the monetary policy from part A), use the government budget constraint
    to calculate how government debt evolves over the next 30 years. It is convenient
    to do this in Excel. 
    \\ \\
    C) Now suppose that the bond market in this country puts an upper limit on the amount
    that it is willing to lend to the government. Suppose this upper limit is equal to
    1 --- i.e., 100\% of GDP. This means that government debt is constrained to be less
    than or equal to 1, i.e., $B_t \leq 1$. In which year will this constraint bind?
    \\ \\
    Imagine that we reach the year in which the constraint in part C) starts binding. 
    Some adjustment is needed to balance the government's budget. But what type of 
    adjustment? We will now see that it depends on which institution is stronger, the
    central bank or Congress. Let's consider each case in turn. 
    \\
    D) Suppose the central bank is compeletely independent and that its commitment to no
    inflation is sufficiently strong that Congress cannot force the central bank to abandon
    its policy. Describe the adjustment Congress is forced to make in this case once the
    constraint binds. In particular, calculate the primary surplus, $T_t - G_t$, that 
    Congress will need to set in the period that the constraint binds and in subsequent
    periods (up to period 30). 
    \\ \\
    E) Suppose instead the Congress can push the central bank around and get it to change 
    its monetary policy once the constraint binds. What will inflation be after the government's
    budget hits the constraint under the assumption that Congress doesn't change the primary
    surplus but rather forces the central bank to ``monetize'' its deficit spending after that 
    point? More specifically, calculate the inflation rate in the year the constraint binds and
    in subsequent years up to year 30. 
    
    \pagebreak
    \part
    
    Suppose the central bank is determined not to allow any inflation. How 
    should it set the money supply to achieve this?
    \\ \\
    \solution

    Since prices are perfectly flexible in this problem, to ensure that there
    is no inflation, the central bank should not change the money supply over
    time, i.e., \fbox{$M_{t+1} = M_{t}$}. Since $M_t = P_t$, this implies that
    prices won't change over time either, $P_{t+1} = P_t$, and hence no inflation.

    \pagebreak
    \part 

    Suppose that dysfunctional politics in this country implies that Congress 
    refuses to collect enough taxes to pay for the amount of spending it decides 
    to do. (Alternatively, one could say that it decides to spend more than the 
    amount of taxes it decides to collect. Specifically, suppose $G_t = 0.25$, 
    while $T_t= 0.15$. Given the monetary policy from part A), use the government 
    budget constraint to calculate how government debt evolves over the next 30 
    years. 
    \\ \\
    \solution

    If the money supply does not change across time, the government budget constraint
    becomes the following
    \[
        \begin{split}
            G_t + (1 + R_{t-1}) B_{t-1} &= T_t + B_t + \frac{M_t - M_{t-1}}{P_t}
            \\
            G_t + (1 + R_{t-1}) B_{t-1} &= T_t + B_t + \frac{0}{P_t}
            \\
            G_t + (1 + R_{t-1}) B_{t-1} - T_t &= B_t
        \end{split}
    \]
    So the government debt evolves according to $B_t = G_t + (1 + R_{t-1}) B_{t-1} - T_t$
    over the next 30 years, which is plotted below starting at $t=-1$ where $G_t = 0.25$,
    $R_t = 0.05$, and $T_t = 0.15$

    \begin{center}
    \begin{tikzpicture}
        \begin{axis}[
            width=0.9\textwidth, % Make the plot wider
            height=0.5\textwidth,
            xmin=-1,
            xmax=30,
            axis line style={line width=1pt},
            xlabel={Time $t$},
            ylabel={},
            title style={font=\large, align=center},
            xlabel style={font=\large},
            legend style={
                at={(0.5,1.15)}, % Place legend below the plot
                anchor=north,
                legend columns=-1, % Arrange legend in a single row
                font=\small,
                draw=black % Add a border to the legend
            },
            grid=both, % Add gridlines
            minor grid style={dashed, gray!30}, % Style for minor gridlines
            major grid style={solid, gray!60}, % Style for major gridlines
            ]
            \draw[thick, black] (axis cs: 0, \pgfkeysvalueof{/pgfplots/ymin}) -- 
                                (axis cs: 0, \pgfkeysvalueof{/pgfplots/ymax});
            \draw[thick, black] (axis cs: \pgfkeysvalueof{/pgfplots/xmin}, 0) -- 
                                (axis cs: \pgfkeysvalueof{/pgfplots/xmax}, 0);
            
            \addplot[blue, thick, mark=*] 
            table[x=t, y=B, col sep=comma] {debt.csv};
            \addlegendentry{$B_t$}

            % \addplot[red, thick, dashed, domain=-1:30] {1};
            % \addlegendentry{Horizontal line $y = 1$}
        \end{axis}
    \end{tikzpicture}
    \end{center}

    \pagebreak
    \part

    Now suppose that the bond market in this country puts an upper limit on the 
    amount that it is willing to lend to the government. Suppose this upper limit
    is equal to 1 --- i.e., 100\% of GDP. This means that government debt is 
    constrained to be less than or equal to 1, i.e., $B_t \leq 1$. In which year 
    will this constraint bind?
    \\ \\
    \solution

    \begin{center}
    \begin{tikzpicture}
        \begin{axis}[
            width=0.9\textwidth, % Make the plot wider
            height=0.5\textwidth,
            xmin=4,
            xmax=9,
            axis line style={line width=1pt},
            xlabel={Time $t$},
            ylabel={},
            title style={font=\large, align=center},
            xlabel style={font=\large},
            legend style={
                at={(0.5,1.15)}, % Place legend below the plot
                anchor=north,
                legend columns=-1, % Arrange legend in a single row
                font=\small,
                draw=black % Add a border to the legend
            },
            grid=both, % Add gridlines
            minor grid style={dashed, gray!30}, % Style for minor gridlines
            major grid style={solid, gray!60}, % Style for major gridlines
            ]
            \draw[thick, black] (axis cs: 0, \pgfkeysvalueof{/pgfplots/ymin}) -- 
                                (axis cs: 0, \pgfkeysvalueof{/pgfplots/ymax});
            \draw[thick, black] (axis cs: \pgfkeysvalueof{/pgfplots/xmin}, 0) -- 
                                (axis cs: \pgfkeysvalueof{/pgfplots/xmax}, 0);
            
            \addplot[blue, thick, mark=*] 
            table[x=t, y=B, col sep=comma] {debt.csv};
            \addlegendentry{$B_t$}

            \addplot[red, thick, dashed, domain=-1:30] {1};
            \addlegendentry{$B_t =1 $}
        \end{axis}
    \end{tikzpicture}
    \end{center}

    I've replotted above an enhanced version of the graph from the last part
    with a horizontal line at the $B_t=1$ level, denoting  the upper limit for 
    the bond market. From the plot, it appears the upper limit is crossed after
    \fbox{year 7} going into year 8.

    \pagebreak
    \part

    Suppose the central bank is compeletely independent and that its commitment to 
    no inflation is sufficiently strong that Congress cannot force the central bank
    to abandon its policy. Describe the adjustment Congress is forced to make in this
    case once the constraint binds. In particular, calculate the primary surplus, 
    $T_t - G_t$, that Congress will need to set in the period that the constraint 
    binds and in subsequent periods (up to period 30). 
    \\ \\
    \solution

    If Congress cannot force the central bank to abandon its policy, the government
    will have to run a surplus such that $B_{t} = B_{t-1}$. I'll solve for the value
    of $T_t - G_t$ that satisfies this constraint
    \[
        \begin{split}
            G_t + (1 + R_{t-1}) B_{t-1} - T_t &= B_t
            \\
            G_t - T_t &= B_t - (1 + R_{t-1}) B_{t-1}
            \\
            G_t - T_t &= B_t - (1 + R_{t-1}) B_{t}
            \\
            G_t - T_t &= B_t (1 - (1 + R_{t-1})) 
            \\
            G_t - T_t &= -B_t R_{t-1}
            \\
            T_t - G_t &= B_t R_{t-1}
        \end{split}
    \]

    The primary surplus required going forward is the product of the debt and the
    interest rate for each period, \fbox{$T_t - G_t = B_t R_{t-1}$}. 

    \pagebreak
    \part

    Suppose instead the Congress can push the central bank around and get it to 
    change its monetary policy once the constraint binds. What will inflation be 
    after the government's budget hits the constraint under the assumption that 
    Congress doesn't change the primary surplus but rather forces the central 
    bank to ``monetize'' its deficit spending after that  point? More specifically, 
    calculate the inflation rate in the year the constraint binds and in subsequent
    years up to year 30. 
    \\ \\
    \solution

    If the central bank is forced to monitize the debt, that means that the money
    supply needs to change in a way so that $(M_t - M_{t-1})/P_t$ balances the
    government budget constraint so that $B_t = B_{t-1}$. Since prices are perfectly
    flexible in this problem, $M_t = P_t$, we can rewrite this real money supply
    change as inflation since $\pi_t = (P_t - P_{t-1})/P_t$

    \[
        \begin{split}
            G_t + (1 + R_{t-1}) B_{t-1} &= T_t + B_t + \frac{M_t - M_{t-1}}{P_t}
            \\
            G_t + (1 + R_{t-1}) B_{t} &= T_t + B_t + \frac{P_t - P_{t-1}}{P_t}
            \\
            G_t + (1 + R_{t-1}) B_{t} - B_t - T_t &= \pi_t
            \\
            G_t + B_t ((1 + R_{t-1}) - 1)  - T_t &= \pi_t
            \\
            G_t + B_t R_{t-1} - T_t &= \pi_t
        \end{split}
    \]

    Now substituing in values for $G_t$, $R_t$, and $T_t$

    \[
        \begin{split}
            G_t + B_t R_{t-1} - T_t &= \pi_t
            \\
            0.25 + 0.05B_t - 0.15 &= \pi_t
            \\
            0.10 + 0.05B_t &= \pi_t
        \end{split}
    \]

    So inflation will equal \fbox{$\pi_t = 0.10 + 0.05B_t$}.
    
\end{homeworkProblem}